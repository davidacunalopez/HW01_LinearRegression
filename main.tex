% !TeX program = pdflatex
\documentclass[conference]{IEEEtran}
\IEEEoverridecommandlockouts

% ---- Paquetes ----
\usepackage[spanish]{babel}
\usepackage[utf8]{inputenc}
\usepackage[T1]{fontenc}
\usepackage{amsmath, amssymb}
\usepackage{booktabs}
\usepackage{siunitx}
\usepackage{graphicx}
\usepackage{array}
\usepackage{hyperref}
\usepackage{multirow}
\usepackage{caption}
\usepackage{subcaption}
\usepackage{float}
\sisetup{detect-weight=true,detect-family=true}

% Rutas comunes para figuras
\graphicspath{{figs/}{figures/}{images/}{img/}{./}}

% --- Utilidades de placeholders de figuras ---
\newcommand{\placeholderbox}[2][5cm]{%
  \fbox{\begin{minipage}[c][#1][c]{0.9\linewidth}\centering \textbf{Placeholder:} #2\end{minipage}}}

% label, caption, filename
\newcommand{\placefigure}[3]{%
\begin{figure}[H]\centering
\IfFileExists{#3}{\includegraphics[width=\linewidth]{#3}}{\placeholderbox{#3}}
\caption{#2}\label{#1}\end{figure}}

% width, filename, subcaption, label
\newcommand{\placesubfig}[4]{%
\begin{subfigure}{#1}
\centering
\IfFileExists{#2}{\includegraphics[width=\linewidth]{#2}}{\placeholderbox{#2}}
\caption{#3}\label{#4}
\end{subfigure}}

% ---- Título y autores ----
\title{Implementación de Regresión Lineal
Manual con Descenso de Gradiente\vspace{-0.35em}}

\author{\IEEEauthorblockN{Priscilla Jim\'enez Salgado\IEEEauthorrefmark{1}, Fabi\'an Araya Ortega\IEEEauthorrefmark{1}, David Acu\~na L\'opez\IEEEauthorrefmark{1}}
\IEEEauthorblockA{\IEEEauthorrefmark{1}Curso de Inteligencia Artificial, Escuela de Ingenier\'ia en Computaci\'on, Tecnol\'ogico de Costa Rica (TEC)\\}}

\begin{document}
\maketitle

\begin{abstract}
Se presenta un informe t\'ecnico del estudio de rendimiento acad\'emico estudiantil basado en un conjunto de 10{,}000 observaciones y 6 variables (5 num\'ericas y 1 categ\'orica). Tras una limpieza que elimina 127 duplicados exactos, el conjunto final contiene 9{,}873 filas. El an\'alisis exploratorio (EDA) confirma ausencia de nulos y de \\emph{outliers} relevantes (ajustando el m\'etodo IQR a rangos reales). La correlaci\'on m\'as alta con el rendimiento (\emph{Performance Index}) es la de \emph{Previous Scores} ($r\approx0{.}915$), seguida de \emph{Hours Studied} ($r\approx0{.}374$); el resto de variables muestran asociaciones d\'ebiles. Para evitar fuga de informaci\'on, la divisi\'on en train/val/test (70/15/15) se realiza por grupos de vectores de caracter\'isticas id\'enticos. Se implementa regresi\'on lineal con descenso por gradiente (batch y mini-batch); los errores en test son \SI{4.3302}{} (batch) y \SI{4.3518}{} (mini-batch), con $R^2\approx0{.}988$.\footnote{Este informe resume \'unicamente los resultados cuantitativos y hallazgos documentados en el PDF del cuaderno de trabajo.}
\end{abstract}

\begin{IEEEkeywords}
EDA, regresi\'on lineal, descenso por gradiente, muestreo estratificado, fuga de informaci\'on, rendimiento acad\'emico.
\end{IEEEkeywords}

\section{Introducci\'on}
Este documento reporta los hallazgos de un estudio de predicci\'on del rendimiento acad\'emico estudiantil. El flujo metodol\'ogico comprende: (i) revisi\'on estructural del conjunto de datos, (ii) EDA con estad\'isticas y gr\'aficos, (iii) control de duplicados y definici\'on de grupos de caracter\'isticas, (iv) divisi\'on del conjunto en entrenamiento/validaci\'on/prueba mediante muestreo aleatorio y estratificado por cuartiles del objetivo, y (v) ajuste y evaluaci\'on de un modelo de regresi\'on lineal optimizado por descenso del gradiente (batch y mini-batch).

\section{Conjunto de Datos y Preprocesamiento}
\subsection{Estructura y tipos}
El conjunto tiene 6 columnas: \emph{Hours Studied}, \emph{Previous Scores}, \emph{Extracurricular Activities} (categ\'orica Yes/No), \emph{Sleep Hours}, \emph{Sample Question Papers Practiced} y \emph{Performance Index}. Se verifica ausencia de valores nulos y tipos consistentes para an\'alisis num\'erico.

\subsection{Limpieza y duplicados}
Se eliminan 127 filas duplicadas exactas, pasando de 10{,}000 a 9{,}873 instancias. Se identifican 1{,}312 filas que comparten exactamente el mismo vector de caracter\'isticas (\(\mathbf{x}\) id\'entico) pero con potenciales diferencias en la variable objetivo; estos casos no se eliminan y se tratan en la etapa de divisi\'on para prevenir fuga de informaci\'on.

\subsection{Codificaci\'on}
Se crea una versi\'on binaria de \emph{Extracurricular Activities} (Yes=1, No=0) para c\'alculo de correlaciones y modelado.

\section{An\'alisis Exploratorio de Datos (EDA)}


% --- Figuras EDA: distribuciones univariadas ---
\begin{figure}[H]\centering
  \placesubfig{0.48\linewidth}{eda_hours_hist.png}{Histograma: Hours Studied}{fig:hist_hours}
  \placesubfig{0.48\linewidth}{eda_prev_scores_hist.png}{Histograma: Previous Scores}{fig:hist_prev}
  \placesubfig{0.48\linewidth}{eda_sleep_hist.png}{Histograma: Sleep Hours}{fig:hist_sleep}
  \placesubfig{0.48\linewidth}{eda_sample_papers_hist.png}{Histograma: Sample Papers}{fig:hist_papers}
  \caption{Distribuciones univariadas (ejemplos).}
  \label{fig:eda_dists}
\end{figure}

Las variables num\'ericas presentan rangos plausibles (por ejemplo, \emph{Hours Studied} en [1,9], \emph{Previous Scores} en [40,99], \emph{Sleep Hours} en [4,9], \emph{Sample Question Papers Practiced} en [0,9], y \emph{Performance Index} en [10,100]). Media y mediana son coherentes, sin indicios de sesgos extremos.


% --- Figuras EDA: boxplots ---
\placefigure{fig:boxplots}{Boxplots por variable (verificaci\'on de outliers).}{eda_boxplots.png}

Aplicando IQR con l\'imites te\'oricos ajustados a los rangos observados de cada variable, no se detectan outliers relevantes; por tanto, no se aplican recortes ni transformaciones especiales.


% --- Figuras EDA: correlaciones y dispersi\'on ---
\begin{figure}[H]\centering
  \placesubfig{0.48\linewidth}{corr_heatmap.png}{Matriz de correlaci\'on}{fig:corr_heatmap}
  \placesubfig{0.48\linewidth}{target_scatter_prev_scores.png}{Objetivo vs Previous Scores}{fig:scat_prev}
  \caption{Relaciones con la variable objetivo.}
  \label{fig:corr_and_scatter}
\end{figure}

La Tabla~\ref{tab:corr} resume las correlaciones de Pearson con \emph{Performance Index}: \emph{Previous Scores} ($r\approx0{.}915$) y \emph{Hours Studied} ($r\approx0{.}374$) muestran asociaciones significativa y moderada, respectivamente; \emph{Sleep Hours} ($r\approx0{.}048$), \emph{Sample Papers Practiced} ($r\approx0{.}043$) y la versi\'on binaria de \emph{Extracurricular} ($r\approx0{.}025$) presentan relaciones d\'ebiles.

\begin{table}[H]
  \centering
  \caption{Correlaci\'on de cada variable con el objetivo (\emph{Performance Index}).}
  \label{tab:corr}
  \begin{tabular}{lS[table-format=1.3]}
    \toprule
    Variable & {\(r\)} \\
    \midrule
    Previous Scores & 0.915 \\
    Hours Studied & 0.374 \\
    Sleep Hours & 0.048 \\
    Sample Question Papers Practiced & 0.043 \\
    Extracurricular (bin) & 0.025 \\
    \bottomrule
  \end{tabular}
\end{table}

\section{Divisi\'on del Conjunto de Datos}
Para evitar fuga de informaci\'on, la unidad de muestreo es el \emph{grupo de caracter\'isticas id\'enticas}. Se crean identificadores de grupo, y luego se dividen los grupos en proporciones 70/15/15.



% --- Divisi\'on: tama\~nos de particiones (aleatorio por grupos) ---
\placefigure{fig:split_random}{Tama\~no de particiones (aleatorio por grupos).}{split_random_bars.png}

Se reporta: 9{,}208 grupos \'unicos; filas por grupo en $[1,3]$. Asignaci\'on: 6{,}445 grupos a entrenamiento, 1{,}381 a validaci\'on y 1{,}382 a prueba, resultando en 6{,}930 (70.2\%), 1{,}474 (14.9\%) y 1{,}469 (14.9\%) filas, respectivamente, sin solapamientos entre particiones.


% --- Divisi\'on: estratificado ---
\begin{figure}[H]\centering
  \placesubfig{0.48\linewidth}{split_strat_counts.png}{Tama\~nos por partici\'on}{fig:strat_counts}
  \placesubfig{0.48\linewidth}{strata_distributions.png}{Distribuciones por estrato}{fig:strata_dist}
  \caption{Muestreo estratificado por cuartiles del objetivo.}
  \label{fig:stratified}
\end{figure}

Se definen estratos por cuartiles de \emph{Performance Index} (cuantiles aproximados: Q1=40, Q2=55, Q3=70). La asignaci\'on final por filas es: 6{,}892 (69.8\%) entrenamiento, 1{,}492 (15.1\%) validaci\'on y 1{,}489 (15.1\%) prueba, preservando la distribuci\'on por estratos en cada subconjunto y sin solapamientos.

\section{Modelo de Regresi\'on Lineal}
\subsection{Formulaci\'on}
Se ajusta un modelo lineal $\hat{y}=\mathbf{x}^{\top}\boldsymbol{\theta}$ minimizando el error cuadr\'atico medio (MSE). La actualizaci\'on por descenso del gradiente (batch) est\'andar es
\begin{equation}
  \boldsymbol{\theta} \leftarrow \boldsymbol{\theta} - \alpha\,\nabla_{\boldsymbol{\theta}}\,\mathcal{L}(\boldsymbol{\theta}),
\end{equation}
con tasa de aprendizaje $\alpha$ y funci\'on de p\'erdida $\mathcal{L}$ definida como el MSE.


% --- Curvas de aprendizaje ---
\begin{figure}[H]\centering
  \placesubfig{0.48\linewidth}{loss_train_val_batch.png}{MSE train/val (batch GD)}{fig:loss_batch}
  \placesubfig{0.48\linewidth}{loss_train_val_minibatch.png}{MSE train/val (mini-batch GD)}{fig:loss_minibatch}
  \caption{Curvas de aprendizaje durante el entrenamiento.}
  \label{fig:learning_curves}
\end{figure}

\subsection{M\'etricas finales}

\section{Discusi\'on}

\section{Conclusiones}

\section{Bibliograf\'ia}



\end{document}
